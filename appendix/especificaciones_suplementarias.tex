\chapter{Especificaciones suplementarias}\label{apendice.B}
\section{Introducción}
El propósito de este documento es definir los requerimientos del Sistema web de aprendizaje colaborativo que no pueden ser capturados vía especificación de casos de uso. Estos requerimientos que en su mayoría corresponden a requerimientos no funcionales serán capturados y clasificados según la norma ISO-9126 que define atributos de calidad para productos software.
\section{Funcionalidad} 
\begin{itemize}
\item El sistema no debe tener errores en la asignación de tiempos para cada reunión.
\item El sistema debe validar correctamente la asignación de alumnos a los grupos expertos.
\item El sistema debe limitar el acceso sólo a usuarios debidamente registrados.
\end{itemize}
\section{Fiabilidad o confiabilidad} 
\begin{itemize}
\item El sistema debe funcionar las 24 horas del día.
\item El sistema debe soportar una concurrencia de hasta al menos 50 usuarios.
\item El sistema debe capturar cualquier error y notificarlo sin tener que dejar de funcionar.
\end{itemize}
\section{Usabilidad} 
\begin{itemize}
\item El sistema debe ser intuitivo para el usuario.
\item El sistema debe permitir al usuario que éste aprenda cómo usarlo sin ninguna ayuda adicional.
\item El sistema debe ser amigable para el usuario en cuanto a interfaz(colores y tonos claros, letras claras y legibles).
\end{itemize}
\section{Eficiencia o Performance} 
\begin{itemize}
\item El tiempo de respuesta del sistema no debe exceder de 2 segundos.
\end{itemize}
\section{Mantenibilidad} 
\begin{itemize}
\item El sistema web debe permitir realizar actividades correctivas o de actualización sin afectar el correcto funcionamiento de sus distintos módulos.
\end{itemize}
\section{Portabilidad} 
\begin{itemize}
\item El sistema debe funcionar en los navegadores Chrome y Mozilla Firefox en sus últimas versiones.
\item El sistema debe ser multiplataforma.
\end{itemize}

