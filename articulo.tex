%%%%%%%%%%%%%%%%%%%%%%%%%%%%%%%%%%%%%%%%%
% Journal Article
% LaTeX Template
% Version 1.3 (9/9/13)
%
% This template has been downloaded from:
% http://www.LaTeXTemplates.com
%
% Original author:
% Frits Wenneker (http://www.howtotex.com)
%
% License:
% CC BY-NC-SA 3.0 (http://creativecommons.org/licenses/by-nc-sa/3.0/)
%
%%%%%%%%%%%%%%%%%%%%%%%%%%%%%%%%%%%%%%%%%

%----------------------------------------------------------------------------------------
%	PACKAGES AND OTHER DOCUMENT CONFIGURATIONS
%----------------------------------------------------------------------------------------

\documentclass[twoside]{article}

%\usepackage{lipsum} % Package to generate dummy text throughout this template

\usepackage[utf8]{inputenc} % acentos sin codigo
\usepackage[spanish, es-tabla]{babel} % espanol


\usepackage[sc]{mathpazo} % Use the Palatino font
\usepackage[T1]{fontenc} % Use 8-bit encoding that has 256 glyphs
\linespread{1.05} % Line spacing - Palatino needs more space between lines
\usepackage{microtype} % Slightly tweak font spacing for aesthetics

\usepackage[hmarginratio=1:1,top=32mm,columnsep=20pt]{geometry} % Document margins
\usepackage{multicol} % Used for the two-column layout of the document
\usepackage[hang, small,labelfont=bf,up,textfont=it,up]{caption} % Custom captions under/above floats in tables or figures
\usepackage{booktabs} % Horizontal rules in tables
\usepackage{float} % Required for tables and figures in the multi-column environment - they need to be placed in specific locations with the [H] (e.g. \begin{table}[H])
%\usepackage{hyperref} % For hyperlinks in the PDF
\usepackage[colorlinks=true,linkcolor=blue]{hyperref}

\usepackage{lettrine} % The lettrine is the first enlarged letter at the beginning of the text
\usepackage{paralist} % Used for the compactitem environment which makes bullet points with less space between them

\usepackage{abstract} % Allows abstract customization
\renewcommand{\abstractnamefont}{\normalfont\bfseries} % Set the "Abstract" text to bold
\renewcommand{\abstracttextfont}{\normalfont\small\itshape} % Set the abstract itself to small italic text

\usepackage{titlesec} % Allows customization of titles
\renewcommand\thesection{\Roman{section}} % Roman numerals for the sections
\renewcommand\thesubsection{\Roman{subsection}} % Roman numerals for subsections
\titleformat{\section}[block]{\large\scshape\centering}{\thesection.}{1em}{} % Change the look of the section titles
\titleformat{\subsection}[block]{\large}{\thesubsection.}{1em}{} % Change the look of the section titles

\usepackage{fancyhdr} % Headers and footers
\pagestyle{fancy} % All pages have headers and footers
\fancyhead{} % Blank out the default header
\fancyfoot{} % Blank out the default footer
%\fancyhead[C]{Sistema Jigsaw Coding $\bullet$ \today $\bullet$ Vol. XXI, No. 1} %
\fancyhead[C]{Sistema Jigsaw Coding $\bullet$ \today $\bullet$ UNMSM} % Custom header text
\fancyfoot[RO,LE]{\thepage} % Custom footer text

\usepackage[apaciteclassic,nodoi]{apacite}

%----------------------------------------------------------------------------------------
%	TITLE SECTION
%----------------------------------------------------------------------------------------

\title{\vspace{-15mm}\fontsize{24pt}{10pt}\selectfont\textbf{Sistema web para la enseñanza de algoritmos y programación usando Jigsaw}} % Article title

\author{
\large
\textsc{Leibnitz Rojas}\\[2mm] % Your name
\normalsize Univesidad Nacional Mayor de San Marcos \\ % Your institution
\normalsize \href{mailto:leiparov@gmail.com}{leiparov@gmail.com} % Your email address
\vspace{-5mm}
}
\date{}

%----------------------------------------------------------------------------------------

\begin{document}

\maketitle % Insert title

\thispagestyle{fancy} % All pages have headers and footers

%----------------------------------------------------------------------------------------
%	ABSTRACT
%----------------------------------------------------------------------------------------

\begin{abstract}

%\noindent \lipsum[1] % Dummy abstract text
Los estudios muestran que en muchas universidades del mundo, aún existen problemas cuando se trata de enseñar cursos relacionados a programación y algoritmos. Muchos estudiantes repiten las materias y otros simplemente abandonan en mitad de semestre. Por otro lado, existen muchas investigaciones respecto a cómo mejorar los problemas de aprendizaje de los estudiantes y no necesariamente en temas de programación. Muchos autores han aplicado diversas técnicas de aprendizaje colaborativo obteniendo resultados notables en sus alumnos. El objetivo del presente trabajo es desarrollar un sistema web para la enseñanza de algoritmos y programación a través de una técnica de aprendizaje colaborativo, el mismo que permitirá a los estudiantes desarrollar ejercicios y problemas de forma colaborativa.
\end{abstract}

%----------------------------------------------------------------------------------------
%	ARTICLE CONTENTS
%----------------------------------------------------------------------------------------

\begin{multicols}{2} % Two-column layout throughout the main article text

\section{Introducción}

A pesar de que la programación es el corazón de las ciencias de la computación, y por ende, la mayoría de las carreras de computación tienen cursos de programación, los resultados son desalentadores pues existen muchos estudios multi institucionales que indican que hay serias deficiencias en el aprendizaje de alumnos que han pasado uno o más cursos de programación \cite{mccracken_multi-national_2001,lister_multi-national_2004,Tenenberg_studentsdesigning_2005}. Algunas instituciones han logrado mejorar los cursos de programación adoptando el Python como primer lenguaje de programación. Así lo indica \citeA{nikula_python_2007}.\\

Según \citeA{knobelsdorf_teaching_2014}, los altos ratios de fracasos en los cursos de introducción a la teoría de las ciencias de la computación son un problema comun en las universidades de Alemania, Europa, y NorteAmérica, pues los alumnos tiene dificultades con lo contenidos que por naturaleza son abstractos y teóricos. \cite{knobelsdorf_teaching_2014} plantean en su investigación ciertas modificaciones a la pedagogía de un curso dictado en la Universidad de Postdam, Alemania, las mismas que fueron motivadas por un enfoque de aprendizaje cognitivo.\\

Existen diversas técnicas para desarrollar el aprendizaje colaborativo en un aula de clase y una de ellas, muy conocida, es la técnica de Jigsaw o técnica de Rompecabezas. Esta técnica fue creada en (1978) por Aronson et al. y actualmente es una de las más importantes para fomentar la cooperación y discusión entre miembros de una comunidad de aprendizaje y es usada frecuentemente en ambientes face-to-face y en situaciones de aprendizaje en línea \cite{blocher_increasing_2005}.\\

%------------------------------------------------

\section{Estado del arte}



%------------------------------------------------

\section{Aporte práctico}


%------------------------------------------------

\section{Resultados}


%----------------------------------------------------------------------------------------
%	REFERENCE LIST
%----------------------------------------------------------------------------------------
\bibliographystyle{apacite}
\bibliography{referencias}
%\begin{thebibliography}{99} % Bibliography - this is intentionally simple in this template
%
% 
%\end{thebibliography}

%----------------------------------------------------------------------------------------

\end{multicols}

\end{document}
