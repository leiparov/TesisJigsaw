\newpage
\begin{center}
    Leibnitz Pavel Rojas Bustamante
\end{center}
\vspace*{3cm}
\begin{Large}
\textbf{SISTEMA WEB DE TIEMPO REAL PARA EL APRENDIZAJE COLABORATIVO} \\
\end{Large}
\vspace*{5cm}
\begin{flushright}
    \begin{minipage}{.5\textwidth}
    ``Tesis presentada a la Universidad Nacional Mayor de San Marcos, Lima, Perú, para obtener el Título de Ingeniero de Software''.
    \end{minipage}
    \end{flushright}
\vspace*{3cm}
\begin{flushright}
    \begin{minipage}{.5\textwidth}
    Asesora: Mg. Lenis Wong Portillo
    \end{minipage}
\end{flushright}
\vspace*{3cm}
\begin{center}
    UNMSM - LIMA\\
    Julio - 2014
\end{center}

%%COPYRIGHT
\newpage
\vspace*{\fill}
\begin{center}
%\vspace*{\vfill}
\copyright \hspace{0.2cm}Leibnitz Rojas, 2014.\\
Todos los derechos reservados.
\end{center}

%%DEDICATORIA
\newpage
\clearpage
\vspace*{\fill}
\begin{flushright}
\begin{minipage}{.5\textwidth}
Este trabajo esta dedicado a mis padres Arturo y María.\\
\end{minipage}
\end{flushright}
\vfill % equivalent to \vspace{\fill}
%\clearpage
\newpage
\chapter*{Agradecimientos}
%A la profesora Lenis Wong, por su orientación y dedicación para que este trabajo cumpla con los objetivos trazados.

A la profesora Lenis Wong por su orientación, consejos  y revisiones del presente trabajo.\\

A mis colegas y amigos de la escuela de Ingeniería de Software por sus observaciones y porque en todo momento me incentivaron para que culmine este trabajo.\\

A los profesores de la UNMSM, principalmente a la profesora Lenis Wong por sus observaciones teóricas que me sirvieron de mucho.\\

A todas aquellas personas que indirectamente me ayudaron para culminar este trabajo y que muchas veces constituyen un invalorable apoyo.\\

\chapter*{Resumen}

Los estudios muestran que en muchas universidades del mundo, aún existen problemas cuando se trata de enseñar cursos relacionados a programación y algoritmos. Muchos estudiantes repiten las materias y otros simplemente abandonan en mitad de semestre.\\

Existen muchas investigaciones respecto a cómo mejorar los problemas de aprendizaje de los estudiantes y no necesariamente en temas de programación. Muchos autores han aplicado diversas técnicas de aprendizaje colaborativo obteniendo resultados notables en sus alumnos.\\

El objetivo del presente trabajo es desarrollar un sistema web para la enseñanza de algorirmos y programación, el mismo que permitirá a los estudiantes desarrollar temas a través de técnicas de aprendizaje colaborativo como la técnica de Jigsaw y la técnica de Investigación en pares.
%Se deberá explicar la problemática del tema de tesis, la misma que deberá ser justificada desde el punto de vista teórico y práctico (ver sección de justificación). Seguidamente, se deberá exponer el aporte teórico - práctico, indicando los beneficios de la propuesta. Deberá mencionar brevemente los grandes temas del trabajo y finalizar con la principal conclusión del trabajo.
%Palabras Claves: colocar, entre comas, de tres a cinco palabras

\chapter*{Abstract}
Studies show that in many universities around the world, there are still problems when it comes to teaching courses related to programming and algorithms. Many students repeat the courses and others just leave in the middle of the semester. \\

There are many studies on how to improve the learning problems of students and not necessarily on programming topics. Many authors have applied various techniques of collaborative learning obtaining remarkable results in their students. \\

The aim of this work is to develop a web system for teaching programming and algorithms, the same will enable students to develop issues through collaborative learning techniques like Jigsaw and Pair Research.
