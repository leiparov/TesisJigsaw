
\chapter{Introducción}
\section{Antecedentes}
La idea del aprendizaje colaborativo(AC) empezó a ser de interés para los profesores de colegios americanos allá por el año 1980, pero la primera idea básica fue desarrollada en los años 1950 a 1960 por un grupo de profesores e investigadores británicos \cite{bruffee_collaborative_1984}. Después de estudiar la interacción entre estudiantes de medicina y su profesores de física, M.L.J Abercrombie concluyó que los estudiantes de medicina que aprendieron a realizar diagnósticos como un grupo alcanzaron un buen juicio médico, más rápido que aquellos que trabajaron individualmente. Bruffee además plantea que su primer encuentro con la creencia de AC fue cuando se encontró con las conclusiones de un grupo de investigadores que pensaban que el AC se deriva de un ataque contra los estilos de enseñanza autoritarios.\\

El aprendizaje basado en proyectos colaborativos con equipos distribuidos está siendo revolucionado por los rápidos avances tecnológicos que existen hoy en día. Tanto profesores, alumnos e información para las clases deben ser reunidas en un entorno virtual para reducir las barreras geográficas y temporales de cada uno de los miembros de cada equipo de aprendizaje \cite{wang_computer-supported_2002}. Wei Wang, en el 2002 propuso en su tesis Computer-Supported Virtual Collaborative Learning and Assessment Framework for Distributed Learning Environment un marco de trabajo para el aprendizaje cooperativo de equipos distribuidos y con ello diseñó e implementó un Sistema de Soporte para la Enseñanza y Aprendizaje Colaborativo (CLASS por sus siglas en inglés: Collaborative Learning Assessment Support System).\\

Así mismo, en el año 2010, en la Universidad Pinar del Río se vió la necesidad de elaborar una herramienta de software que sirviera de apoyo para la enseñanza del curso de Inteligencia Artificial y así, Salao Bravo, J. R en su tesis “Estudio de las técnicas de Inteligencia Artificial mediante el apoyo de un software Educativo” implementó un sistema web con el objetivo de potenciar el estudio de las técnicas, funcionamiento y aplicaciones de la Inteligencia Artificial \cite{salao_bravo_estudio_2010}.\\

Según \cite{laal_collaborative_2012}, el aprendizaje colaborativo es un enfoque educacional de enseñanza y aprendizaje que involucra grupos de estudiantes trabajando juntos para resolver un problema, completar una tarea, o crear un producto y también significa aprender a través del trabajo en conjunto en lugar de aprender por uno mismo \cite{barkley_collaborative_2012}.\\

El aprendizaje cooperativo o aprendizaje colaborativo es una técnica de enseñanza muy conocida y que se ha aplicado con una gran variedad de materias y un amplio espectro de las poblaciones \cite{beck_experimental_2008}.\\

\cite{azizinezhad_application_2013} realizaron un estudio para investigar los efectos del aprendizaje colaborativo en el aprendizaje del idioma inglés como lengua extranjera para los alumnos. En dicho estudio se concluyó que los alumnos fueron capaces de mostrar mejores y significativas competencias lingüísticas, competencia discursiva, competencias estratégicas y competencias de comunicación no verbal que el resto de alumnos. En un entorno de aprendizaje cooperativo, hubo muchas tareas interactivas, que de forma natural, estimularon las habilidades sociales, lingüísticas y cognitivas de los estudiantes. Las actividades cooperativas tendían a integrar la adquisición de aquellas habilidades, y crear potentes oportunidades de aprendizaje.\\

Existen diversas técnicas para desarrollar el aprendizaje colaborativo en un aula de clase y una de ellas, muy conocida, es la técnica de Jigsaw. Esta técnica fue creada en (1978) por Aronson et al. y actualmente es una de las más importantes técnicas para fomentar la cooperación y discusión entre miembros de una comunidad de aprendizaje y es usada frecuentemente en ambientes face-to-face y en situaciones de aprendizaje en línea \cite{blocher_increasing_2005}. De acuerdo con \cite{aronson_jigsaw_1978}, usualmente en un Jigsaw el contenido se divide en 5 a 6 subtemas y a cada alumno se le asigna la tarea de estudiar a detalle su respectivo subtema. Los alumnos repasan en grupo el subtema para convertirse en “expertos”. Al final de esta fase, los grupos de expertos se dispersan y se forman nuevos grupos llamados “grupos jigsaw o grupos rompecabezas”. Dentro del nuevo grupo, a cada alumno se le pide que informe sobre su subtema a los demás, y así, al final, todos los grupos obtienen una visión completa de los contenidos.\\

	Según los creadores de la técnica Jigsaw \cite{aronson_jigsaw_1978} ,ésta es particularmente apropiada cuando el tópico de estudio es fácil de fragmentar en sub tópicos, y/o en aquellos contextos donde es particularmente importante trabajar sobre la responsabilidad individual. Sin embargo, cuando se diseña un Jigsaw Online, hay aspectos críticos que se deben tomar seriamente en consideración: el tamaño de la población objetivo, las restricciones de tiempo y la necesidad de un sistema de comunicación bien estructurado \cite{persico_pozzi_sarti_2008}.\\

La técnica de rompecabezas o técnica de Jigsaw, fue implementada en un sistema web en el año 2013 en la Universidad Pontificia Católica del Perú con el fin de automatizar los procesos que se requiere aplicar dicha técnica al aprendizaje colaborativo. A través de ese sistema los alumnos pudieron aprender conceptos sobre Casos de Uso de una manera diferente a una clase tradicional \cite{pinzas_desarrollo_2013}.\\

	La técnica de Jigsaw ha sido usada en los procesos educacionales en países de todos los continentes. Este método puede mejorar el rendimiento de los alumnos y estudiantes a través del aprendizaje colaborativo, así lo afirma  \cite{maftei_strengthen_2011}. Así mismo, \cite{kilic_the_ffect_2008} sostiene que el aprendizaje colaborativo es el proceso de aprendizaje de aquellos que no conocen mucho sobre un tema trabajando en conjunto con aquellos que sí lo conocen, y esto es un concepto que continuamente atrae a muchos docentes; Según Kilic, el aprendizahe colaborativo es un proceso que se enfoca en desarrollar a los estudiantes social e intelectualmente. Además, varias investigaciones han mostrado que especialmente en primaria, secundaria y universidad, la técnica Jigsaw es efectiva en el proceso de aprendizaje de cursos teóricos, en el desarrollo de pensamiento crítico de los estudiantes y en sus habilidades de comunicación.\\

\section{Definición del Problema}
 Hoy en día, muchos estudiantes tienen dificultades para llevar con éxito los cursos de algoritmos y programación, problema que se evidencia en el porcentaje de alumnos que desaprueban los exámenes, que desaprueban el curso o que simplemente se retiran a mitad de ciclo. Además, hasta ahora, las clases en la Facultad de Ingeniería de Sistemas e Informática(FISI) se vienen impartiendo a través de un enfoque de enseñanza tradicional que dista mucho de promover el aprendizaje colaborativo.

\section{Justificación}

\cite{kinnunen_why_2006} sostiene que los cursos introductorios de programación frecuentemente tienen un alto porcentaje de desaprobados y retiros por parte de los alumnos y a pesar que existen diversos enfoques que han tratado de reducir estos porcentajes y en los cuales se incluyen estrategias de aprendizaje colaborativo como el trabajo en equipos y la instrucción entre pares, muchos estudios multi institucionales \cite{mccracken_multi-national_2001} \cite{lister_multi-national_2004} \cite{Tenenberg_studentsdesigning_2005} han indicado que hay serias deficiencias en el aprendizaje de los estudiantes que han pasado uno o varios cursos de programación.\\

Por otro lado, \cite{cliburn_team-based_2014} desarrolló el curso de Estructura de Datos a través del aprendizaje basado en equipos y el aprendizaje tradicional con el fin de comparar resultados en las evaluaciones de los estudiantes, y, aunque no encontró diferencias significativas entre ambas secciones de alumnos, aún continúa usando el aprendizaje en equipos debido a la alta satisfacción que los alumnos muestran en comparación con el método de enseñanza tradicional.\\

\cite{martinez_cooperative_2011} presentaron el diseño, implementación y evaluación de una estrategia de enseñanza basada en aprendizaje cooperativo para introducir el tema de álgebra relacional en un curso de base de datos. La estrategia fue evaluada desde las perspectiva del alumno y del profesor, y se encontró que entre el $78\%$ y el $92\%$ de los estudiantes consideraron que el trabajo en grupo enriqueció su aprendizaje, dando soporte al uso del aprendizaje colaborativo. 

\section{Objetivos}
\subsection{General}
Desarrollar un sistema web de tiempo real para promover el aprendizaje colaborativo de los estudiantes de la FISI a través de la técnica de Jigsaw y enfocándolo específicamente a la enseñanza de cursos de algoritmos y programación.
\subsection{Específicos}
\begin{itemize}
  \item Investigar y analizar diferentes técnicas para el aprendizaje colaborativo y sus beneficios.
  \item Definir métricas de calidad para el desarrollo del sistema.
  \item Investigar sobre sistemas web y herramientas informáticas que permitan el trabajo en equipo de manera virtual.
  \item Definir el proceso de desarrollo para el sistema a implementar.
  \item Investigar sobre los procesos de implementación de la técnica Jigsaw.
  \item Investigar sobre los contenidos que se dicta en cursos de algoritmos y programación.
\end{itemize}

\section{Alcances}
La presente tesis tendrá los siguientes alcances:
\begin{itemize}
  \item El sistema que se desarrollará permitirá a los docentes de la FISI desarrollar clases usando un enfoque de aprendizaje colaborativo en los estudiantes.
  \item El sistema a implementar se enfocará específicamente en el aprendizaje  colaborativo de temas de algoritmos y programación.
  \item Se realizará el estado del arte de distintas herramientas y técnicas aplicadas en el aprendizaje colaborativo.
  \item Se realizará el estado del arte de las diferentes tecnologías que permite desarrollar un sistema web de tiempo real.
  \item Se realizará el estado del arte de las últimas investigaciones sobre la aplicación de aprendizaje colaborativo en cursos relacionados a la ingeniería de software.
\end{itemize}

\section{Estructura de la Tesis}
La presente tesis está organizada en X capítulos que a continuación se explican brevemente.\\

En el Capítulo 2 se describe el marco teórico, donde se explican los concepto fundamentales sobre Aprendizaje cooperativo, los mismo que son abordados a lo largo de toda esta investigación. Así mismo, también se detallan lo conceptos principales para entender el funcionamiento de sistemas web de tiempo real.\\

En el Capítulo 3 se describe el estado del arte, donde se describen y analizan algunas de las técnicas existentes para el desarrollo del aprendizaje colaborativo en la enseñanza de temas de ingeniería de software.

