\chapter{Marco Teórico}
\section{Aprendizaje Colaborativo}
\subsection{Definición}
Existen diversas formas de definir lo que es Aprendizaje Colaborativo; \citeA{macgregor_collaborative_1990} dice que la enseñanza y el aprendizaje colaborativo son un enfoque educacional que involucra grupos de estudiantes trabajando juntos para resolver un problema, completar una tarea o crear un producto y \citeA{gerlach_1994} sostiene que el aprendizaje cooperativo está basado en la idea de que el aprendizaje es un acto social natural en el cual los participantes conversan entre sí mismos y que es a través de la comunicación y la charla donde realmente ocurre el aprendizaje.\\

El aprendizaje colaborativo es un término para describir una variedad de enfoques educacionales que implican reunir el esfuerzo intelectual de los estudiantes, o estudiantes y profesores juntos. Usualmente los estudiantes están trabajando en grupos de dos o más, buscando entender, solucionar problemas o crear productos. Las actividades de aprendizaje cooperativo son variadas, pero la mayoría se centran en la exploración del estudiante o la aplicación de los materiales de curso, no simplemente en la presentación de un tema por parte del profesor \cite{smith_collaborative_1992};además, el aprendizaje colaborativo tiene como principal característica una estructura que permite a lo estudiantes comunicarse entre sí, y es ahí donde ocurre el aprendizaje\cite{golub1988focus}.

\subsection{Elementos en el aprendizaje colaborativo}
\cite{johnson_1984} plantea 5 elementos básicos en el aprendizaje colaborativo. El aprendizaje colaborativo no es simplemente para los estudiantes el hecho de trabajar en grupo y de acuerdo con su investigación,  un ejercicio de aprendizaje sólo califica como colaborativo si están presentes los siguientes elementos:

\begin{itemize}
  \item \emph{La interdependencia positiva}. Los miembros del equipo están obligados a confiar en los demás para alcanzar un objetivo. Si uno de los miembros del equipo falla al realizar su parte, todos sufren las consecuencias. Los miembros del equipo necesitar creer que están unidos con los demás de una forma que aseguren el éxito en conjunto.
  \item \emph{La interacción ``cara a cara'' o simultánea}. Los miembros del equipo se tienen que ayudar y alentar entre sí para aprender. Ellos deben de explicar qué entendieron y así compartir su conocimiento.
  \item \emph{La responsabilidad individual}. Todos los estudiantes de un grupo son responsables de hacer su parte del trabajo.
  \item \emph{Habilidades sociales}. Los estudiantes deben ser alentados y ayudados a desarrollar y practicar la confianza de equipo, liderazgo, toma de decisiones, comunicación, y manejo de conflictos.
  \item \emph{Autoevaluación de grupo}. Los miembros del equipo tienen que fijarse objetivos, revisar periódicamente qué están haciendo bien como equipo, e identificar cambios por hacer con el fin de mejorar la efectividad a futuro.
\end{itemize}
\subsection{Beneficios del aprendizaje colaborativo}
Numerosos beneficios han sido descritos para el aprendizaje cooperativo \cite{panitz_1999}. Una buena forma de organizarlos es colocándolos en categorías. \citeA{johnsons_1989, panitz_1999} hicieron una lista de más de 50 beneficios para el aprendizaje cooperativo, algunos de los cuales se presentan a continuación:

\begin{enumerate}
  \item Beneficios sociales
  \begin{enumerate}
    \item Ayuda a desarrollar un sistema de apoyo social para los estudiantes.
    \item Lleva a construir un entendimiento de la diversidad entre los estudiantes y el personal.
    \item Establece un entorno positivo para modelar y practicar la cooperación y el trabajo en equipo.
    \item Desarrolla comunidades de aprendizaje.
  \end{enumerate}
  \item Beneficios psicológicos
  \begin{enumerate}
    \item La instrucción centrada en los estudiantes aumenta la autoestima de los mismos.
    \item La cooperación reduce la ansiedad.
    \item El aprendizaje cooperativo desarrolla actitudes positivas hacia los profesores.
  \end{enumerate}
  \item Beneficios académicos
  \begin{enumerate}
    \item El aprendizaje cooperativo promueve habilidades de pensamiento crítico.
    \item Envuelve a los estudiantes activamente en el proceso de aprendizaje.
    \item Los resultados de clase son mejorados.
    \item El aprendizaje cooperativo modela técnicas apropiadas para la resolución de problemas.
    \item Grandes conferencias pueden ser personalizadas.
    \item El aprendizaje es especialmente útil para motivar a los estudiantes en un plan de estudios específico.
  \end{enumerate}
\end{enumerate}

\section{Qué es tiempo real}
El término tiempo real se refiere a la naturaleza oportuna entre la ocurrencia de un evento y el ser advertidos de ello. La medición en el tiempo entre un evento ocurrido y la entrega de ese evento depende en realidad del evento. Si el evento es la aplicación del pie al frenar un auto, entonces el tiempo entre el pie bajando y los frenos que se aplica tiene que ser absolutamente mínimo. Sin embargo, si el evento es el envío de un mensaje de chat en un foro de fútbol y se muestra a los demás usuarios, es poco probable hacer una gran diferencia de unos segundos. En último caso, el evento tiene que ser entregado en un tiempo suficientemente corto. Si te cortas un dedo, no hay retraso entre el corte y el registro de dolor. Esto es tiempo real. Sin embargo, la posibilidad de desarrollar tiempo real no era inicialmente algo fácil. Pero los desarrolladores han llegado a soluciones inteligentes y ``hacks'' para resolver el problema de comunicación entre el servidor y el cliente \cite{lengstorf_realtime_2013}.

\section{Glosario}

\begin{enumerate}
  \item \textbf{Framework}. La palabra inglesa "framework" (marco de trabajo) define, en términos generales, un conjunto estandarizado de conceptos, prácticas y criterios para enfocar un tipo de problemática particular que sirve como referencia, para enfrentar y resolver nuevos problemas de índole similar.
  \item \textbf{Librería}. En informática, una librería es un conjunto de implementaciones funcionales, codificadas en un lenguaje de programación, que ofrece una interfaz bien definida para la funcionalidad que se invoca.
  \item \textbf{Socket}. Designa un concepto abstracto por el cual dos programas (posiblemente situados en computadoras distintas) pueden intercambiar cualquier flujo de datos, generalmente de manera fiable y ordenada.
  \item \textbf{Tiempo real}. Un sistema en tiempo real (STR) es aquel sistema digital que interactúa activamente con un entorno con dinámica conocida en relación con sus entradas, salidas y restricciones temporales, para darle un correcto funcionamiento de acuerdo con los conceptos de predictibilidad, estabilidad, controlabilidad y alcanzabilidad
  \item \textbf{CSCL}. Computer-Supported Collaborative Learning. Aprendizaje colaborativo apoyado por computador.
\end{enumerate}













