\chapter{Marco Teórico}
\section{Aprendizaje Colaborativo}
Una forma de entender en qué consiste el Aprendizaje Colaborativo(AC) es revisando las definiciones presentadas por expertos en el tema, como sigue:
\begin{itemize}
  \item La enseñanza y el aprendizaje colaborativo es un enfoque educacional que involucra grupos de estudiantes trabajando juntos para resolver un problema, completar una tarea o crear un producto \cite{macgregor_collaborative_1990}.
  \item El aprendizaje cooperativo o colaborativo es un término para describir una variedad de enfoques educacionales que implican reunir el esfuerzo intelectual de los estudiantes, o estudiantes y profesores juntos. Usualmente los estudiantes están trabajando en grupos de dos o más, buscando entender, solucionar problemas o crear productos. Las actividades de aprendizaje cooperativo son variadas, pero la mayoría se centran en la exploración del estudiante o la aplicación de los materiales de curso, no simplemente en la presentación de un tema por parte del profesor \cite{smith_collaborative_1992}
  \item El aprendizaje cooperativo está basado en la idea de que el aprendizaje es un acto social natural en el cual los participantes conversan entre sí mismos. Es a través de la comunicación y la charla que ocurre el aprendizaje \cite{gerlach_1994}.
  \item El aprendizaje colaborativo tiene como principal característica una estructura que permite a lo estudiantes comunicarse entre sí, y es ahí donde ocurre el aprendizaje\cite{golub1988focus}.

\end{itemize}

\cite{johnson_1984} plantea 5 elementos básicos en el aprendizaje cooperativo. El aprendizaje cooperativo no es simplemente para los estudiantes el hecho de trabajar en grupo y de acuerdo con su investigación,  un ejercicio de aprendizaje sólo califica como colaborativo si están presentes los siguientes elementos:

\begin{itemize}
  \item \emph{La interdependencia positiva}. Los miembros del equipo están obligados a confiar en los demás para alcanzar un objetivo. Si uno de los miembros del equipo falla al realizar su parte, todos sufren las consecuencias. Los miembros del equipo necesitar creer que están unidos con los demás de una forma que aseguren el éxito en conjunto.
  \item \emph{La interacción ``cara a cara'' o simultánea}. Los miembros del equipo se tienen que ayudar y alentar entre sí para aprender. Ellos deben de explicar qué entendieron y así compartir su conocimiento.
  \item \emph{La responsabilidad individual}. Todos los estudiantes de un grupo son responsables de hacer su parte del trabajo.
  \item \emph{Habilidades sociales}. Los estudiantes deben ser alentados y ayudados a desarrollar y practicar la confianza de equipo, liderazgo, toma de decisiones, comunicación, y manejo de conflictos.
  \item \emph{Autoevaluación de grupo}. Los miembros del equipo tienen que fijarse objetivos, revisar periódicamente qué están haciendo bien como equipo, e identificar cambios por hacer con el fin de mejorar la efectividad a futuro.
\end{itemize}
\subsection{Beneficios del aprendizaje colaborativo}
Numerosos beneficios han sido descritos para el aprendizaje cooperativo \cite{panitz_1999}. Una buena forma de organizarlos es colocándolos en categorías. \citeA{johnsons_1989, panitz_1999} hicieron una lista de más de 50 beneficios para el aprendizaje cooperativo, algunos de los cuales se presentan a continuación:

\begin{enumerate}
  \item Beneficios sociales
  \begin{enumerate}
    \item Aayuda a desarrollar un sistema de apoyo social para los estudiantes.
    \item Lleva a construir un entendimiento de la diversidad entre los estudiantes y el personal.
    \item Establece un entorno positivo para modelar y practicar la cooperación y el trabajo en equipo.
    \item Desarrolla comunidades de aprendizaje.
  \end{enumerate}
  \item Beneficios psicológicos
  \begin{enumerate}
    \item La instrucción centrada en los estudiantes aumenta la autoestima de los mismos.
    \item La cooperación reduce la ansiedad.
    \item El aprendizaje cooperativo desarrolla actitudes positivas hacia los profesores.
  \end{enumerate}
  \item Beneficios académicos
  \begin{enumerate}
    \item El aprendizaje cooperativo promueve habilidades de pensamiento crítico.
    \item Envuelve a los estudiantes activamente en el proceso de aprendizaje.
    \item Los resultados de clase son mejorados.
    \item El aprendizaje cooperativo modela técnicas apropiadas para la resolución de problemas.
    \item Grandes conferencias pueden ser personalizadas.
    \item El aprendizaje es especialmente útil para motivar a los estudiantes en un plan de estudios específico.
  \end{enumerate}
\end{enumerate}














