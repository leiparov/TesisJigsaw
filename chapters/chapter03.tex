\chapter{Estado del Arte}

\section{Técnicas para el aprendizaje colaborativo}

\subsection{Jigsaw}
El estudio que \citeA{evcim_effects_2013} realizaron fue para determinar los efectos de la técnica de Jigsaw en una clase de Inglés en alumnos de preparatoria en la Universidad Abant Izzet Baysal en Turkía. El total de alumnos fue de 48 los cuales fueron separados de tal forma que se obtuvo un grupo experimental y uno de control para poder aplicar la técnica y comparar los resultados. Tanto el grupo experimental como el de control fueron organizados según los resultados de sy examen parcial y prácticas calificadas dadas hasta el momento.\\

El grupo experimental fue dividido en 2 grupos de 6 alumnos para empezar el trabajo. Con el fin de ver los resultados de la técnica de Jigsaw, a este grupo se le asignó desarrollar el tema de Presente Perfecto siguiendo las pautas de la técnica de Jigsaw. Por otro lado, el grupo de control también recibió la clase de Presente Perfecto pero fue en condiciones normales. Al final, ambos grupos fueron evaluados con el mismo examen y luego de analizar los resultados se concluyó que habían diferencias significativas entre los resultados obtenidos entre ambos grupos y que la técnica de Jigsaw tiene un efecto importante en el rendimiento académico de los alumnos.

\subsection{Pair Research}
Es una nueva forma de interacción para incrementar la productividad, el aprendizaje y la colaboración entre los diferentes grupos de investigación \cite{miller_pair_2014}. Los resultados del estudio mostraron que los miembros de los equipos usaron la investigación en pares en diferentes maneras incluyendo programación en pares, lluvia de ideas, recolección de data y análisis.\\

Pair Research es una generalización de la programación en pares. Cada semana, los miembros de los grupos son emparejados guidados por algoritmo de emparejamiento. Cada pareja se reune por una a dos horas de las cuales la mitad del tiempo es dedicado a trabajar en el proyecto de la otra persona. El trabajo puede ser cualquier actividad relacionada a programación, pruebas, diseño, recolección de datos y análisis, lluvia de ideas y asesoría. La siguiente semana, se forman nuevas parejas y el proceso se repite.\\

El objetivo de la técnica de Pair Research es incrementar la colaboración, el aprendizaje y la productividad en los grupos de investigación. Esta técnica ofrece a los miembros la oportunidad de interactuar con muchos compañeros de diversas perspectivas y con diferentes experiencias.


\section{Herramientas para el aprendizaje colaborativo}

\subsection{LearnCS}
\emph{LearnCS!} es un entorno de programación creado específicamente para el uso de estudiantes de primer año de la carrera de ciencas de la computación. Este programa elimina la necesidad de los alumnos de tener que preocuparse por un editor de texto, comandos linux y proceso de compilación. LearnCS! proveee un entorno web en el cual los alumno pueden escribir, ejecutar y depurar programas usando un interfaz familiar y amigable. El compilador de C embebido que posee el sistema permite al alumno ejecutar sus programas con simplemente hacer click en un botón y obtener los resultados de ejecución de su código fuente \cite{lipman_learncs_2014}.\\

En muchos cursos de programación de primer ciclo, el concepto de depurar un programa es enseñado a finales del curso. En cambio, a través de \emph{LearnCS!}, los alumnos pueden establecer puntos de corte en el programa y empezar a depurar paso a paso su código fuente. Además, el sistema brinda la opción de ver la representación de la memoria y así visualizar en detalle el estado de su programa. Cuando se llaman a funciones, los alumnos aprenden cómo los argumentos y variables locales son colocados en la pila de la memoria y cómo las variables son reservadas en memoria \cite{lipman_learncs_2014}.

\begin{figure}[h]
  \centering
  % Requires \usepackage{graphicx}
  \includegraphics[scale=0.5]{figuras/learncs.png}\\
  \caption{LearnCS!}\label{fig:learncs}
\end{figure}

\subsubsection{Diseño e implementación}
LearnCS! fue creado para proporcionar un ambiente de aprendizaje para estudiantes de informática de primer año. Sus principales objetivos son proporcionar asistencia útil al alumno en la construcción de un modelo mental de la máquina nocional de C a través de la visualización detallada de la memoria de \emph{LearnCS!} y sus mensajes de error integrados en las instalaciones de depuración, y para proporcionar que producen consejos que son útiles para el principiante para localizar y corregir errores de sintaxis\cite{lipman_learncs_2014}.\\

LearnCS! se ejecuta en un navegador web y tal como se muesta en la Figura \ref{fig:learncs}, ofrece lo siguiente:

\begin{enumerate}
  \item (Panel superior-izquierdo) Un área de la pantalla dedicada a la edición del programa que se está desarrollando.
  \item (Panel derecho) Una representación de la vista de la memoria.
  \item (Panel inferior) Una zona de ``salida'' que contiene una ``terminal'' para la entrada y salida de texto.
  \item (Panel inferior-derecho) Opcionalmente, como se muestra en la figura \ref{fig:learncs2}, se muestra un área gráfica que permite desarrollar programas más interactivos y más interesantes para los alumnos.
\end{enumerate}

\begin{figure}[h]
  \centering
  % Requires \usepackage{graphicx}
  \includegraphics[scale=0.5]{figuras/learncs2.png}\\
  \caption{LearnCS!}\label{fig:learncs2}
\end{figure}


\subsection{GoogleDocs}

GoogleDocs es un conjunto de herramientas online que permiten elaborar documentos, hojas de cálculo, dibujos y diapositivas de manera colaborativa usando solamente un navegador web. Esta herramienta es un gestor de documentos pues a través de ella se pueden subir a la nube todo tipo de archivos y ordenarlos en carpetas así como compartirlos con otros usuarios. 


\subsubsection{Funcionalidades}

\begin{enumerate}
  \item Crear documentos básicos desde cero. En google docs se pueden realizar diversos tipos de tareas como crear listas con viñetas, añadir tablas, imágenes, entre otras.
  \item Subir archivos ya creados. Google Docs acepta la mayoría de formatos de archivos como DOC, XLS, ODT, CSV, PPT,PDF, etc.
  \item Invitar a otros usuarios a colaborar en un documentos y permitirles ver y modificar el documento.
  \item Colaborar online en tiempo real con todos los usuarios a los que les fue compartido el documento.
\end{enumerate}
\begin{figure}[h]
  \centering
  % Requires \usepackage{graphicx}
  \includegraphics[scale=0.8]{figuras/googledocs.jpg}\\
  \caption{GoogleDocs}\label{fig:googledocs}
\end{figure}


\subsection{Sistema Web para la enseñanza de Casos
de Uso empleando la Técnica de Aprendizaje Cooperativo
de Rompecabezas.}

Este sistema es producto de una tesis de grado implementada en la Pontificia Universidad Católica del Perú. El sistema pretende dar soporte a las tres fases que comprende una clase en la cual se emplea la técnica de Jigsaw para lo cual, se construyeron los módulos de Planificación, Ejecución y Evaluación.\\

El módulo de Planificación permite realizar el diseño de cada sesión de clase. Ahí se plantean los datos de la sesión que serán la base de los objetivos y resultados esperados que permitirán medir el progreso académico de los alumnos.\\

El módulo de Ejecución se encarga de llevar a cabo la ejecución de una sesión de clase basada en la técnica de Jigsaw. Permite el desarrollo paso a paso desde la lectura de materiales, documentos y casos hasta la diagramación de la solución que brinden cada uno de los grupos Jigsaw y Expertos. En este módulo se cuenta con foros de discusión que permiten la comunicación entre los miembros de cada grupo.\\

Por último se tiene el módulo de Evaluación, en el cual se elaboran preguntas y examenes que luego el profesor aplica a sus alumnos. Estos exámenes son calificados manualmente o de forma automática por el propio sistema.

\subsubsection{ARQUITECTURA DEL SISTEMA}

El sistema fue desarrollado usando una arquitectura Modelo-Vista-Controlador. Se usó Java como plataforma de desarrollo y MySQL como motor de base de datos. Acontinuación se indican los frameworks utilizados en el sistema:

\begin{itemize}
  \item Framework J2EE
  \item Struts 2
  \item MyBatis
  \item Librerías AJAX: JQuery, DojoToolKit
\end{itemize}

\subsubsection{MÓDULOS DEL SISTEMA}

El sistema desarrollado consta de 3 módulos que se detallan a continuación:\\

\textbf{Planificación}\\

Este módulo posee las siguientes funcionalidades:
\begin{enumerate}
  \item Crear dinámica de Curso
  \item Consultar información del curso
  \item Consultar configuración de dinámica
  \item Consultar información de dinámica
  \item Actualizar mensaje interno
\end{enumerate}

\textbf{Ejecución}\\

Este módulo se encarga de llevar a cabo el desarrollo de una sesión basada en la técnica de aprendizaje colaborativo de rompecabezas. Permite la lectura de materiales y documentes, hasta la elaboración colaborativa de la solución a los diferentes problemas propuestos en los diferentes grupos Jigsaw y Expertos.\\

El módulo cuenta con herramientas de chat para la comunicación online, así como también de un foro para la discusión de los temas.\\

Este módulo posee las siguientes funcionalidades:

\begin{enumerate}
  \item Actualizar publicación en foro
  \item Ingresar sesión cooperativa
  \item Controlar estado de la dinámica
  \item Elaborar casos de uso,.
\end{enumerate}

\textbf{Evaluación}\\

En esta parte del sistema es donde se elaboran las preguntas y exámenes por parte del docente y posteriormente se aplican a los alumnos. El sistema permite que estos exámenes sean evaluados manualmente o de forma automática.

Este módulo posee las siguientes funcionalidades:
\begin{enumerate}
  \item Consultar configuración de evaluación.
  \item Consultar corrección de evaluación.
  \item Consultar calificación de evaluación.
  \item Crear evaluación.
  \item Rendir evaluación.
  \item Realizar calificación.
\end{enumerate}


\emph{Fuente} \cite{pinzas_desarrollo_2013}

\subsection{CodeBunks}
Es una plataforma que permite codificar y compilar en diferentes lenguajes de programación de forma colaborativa y en tiempo real.
\subsubsection{Funcionalidades}
\begin{enumerate}
  \item Posee un editor colaborativo que soporta 14 lenguajes de programación, tiene coloración de sintaxis según el lenguaje y brinda un indentado inteligente.
  \item Compilar y Ejecutar. La plataforma permite compilar y ejecutar código en Python, Java, C, C++, Ruby, Javascript, entre otros.
  \item Audio y Video Chat.
  \item Code playback. Permite repetir la historia de los cambios realizados en el código.
  \item Equipos. Permite crear equipos de trabajo e invitar a otros usuarios a colaborar en el desarrollo de un programa.
\end{enumerate}
\begin{figure}[h]
  \centering
  % Requires \usepackage{graphicx}
  \includegraphics[scale=0.3]{figuras/codebunk.png}\\
  \caption{Codebunk}\label{fig:codebunk}
\end{figure}
\section{Frameworks para aplicaciones web de tiempo real}

\subsection{GoogleDrive Realtime API}

La API de GoogleDrive en tiempo real ofrece el trabajo colaborativo como un servicio para los archivos en Google Drive a través del uso de las transformaciones operativas. El API es una biblioteca JavaScript que ofrece a los desarrolladores objetos de colaboración, eventos y métodos para la creación de aplicaciones en las cuales se puedan realizar tareas colaborativas.\\

Esta API permite a los desarrolladores diseñar un modelo de datos común que se ve como un modelo local en memoria. Se pueden escribir código para manipular listas, arrays, matrices, maps, y objetos javascript propios del desarrollador. Cada vez que se modifique un modelo de datos, éste automáticamente cambiará para todos los usuarios presentes en el documento.\\

La API está basada en la misma tecnología de colaboración usada por GoogleDocs y por ello, cada vez que un modelo de datos es modificado, el cambio es aplicado inmediatamante a la copia local del documento y luego, la API envía una representación del cambio al servidor de tal forma que el cambio es guardado en el documento y comunicado a los demás colaboradores.\\

La API en tiempo real se encarga de todos los aspectos de la transmisión de datos, el almacenamiento y la resolución de conflictos cuando varios usuarios están editando un archivo. En general, la API nos brinda lo siguiente:

\begin{enumerate}
  \item Funciones para cargar y trabajar documentos en tiempo real.
  \item Objetos construídos colaborativamente(cadenas, listas, y mapas).
  \item La capacidad de crear objetos propios que puedan personalizarse.
  \item Eventos para la detección de cambios en el modelo de colaboración y detección del ingreso o salida de colaboradores.
\end{enumerate}
La API en tiempo real de GoogleDrive proporciona todas las herramientas que se necesitan para crear una aplicación colaborativa que no necesita correr en nuestro propio servidor.


\subsection{Socket.IO}

Socket.IO permite la comunicación basada en eventos bidireccional en tiempo real.
Funciona en todas las plataformas, el navegador o dispositivo, centrándose también en la fiabilidad y la velocidad.
\begin{itemize}
  \item Análisis en tiempo real
  \item Transmisión binaria. A partir de 1.0, es posible enviar cualquier tipo de archivo: imagen, audio, video.
  \item Mensajería instantánea y chat
  \item Colaboración de documentos. Permitir a los usuarios editar simultáneamente un documento y ver los cambios del otro.
\end{itemize}

\subsection{Ideone API - Sphere Engine}
Sphere Engine, antes conocida como Ideone API, permite a los usuarios ejecutar código en múltiples lenguajes de manera online. Ideone es un compilador y una herramienta de depuración que soporta más de 60 lenguajes de programación. Através de la API, los desarrolladores pueden crear sus propias aplicaciones con fines educativos, personales o de negocios. Sphere Engine es un servicio web el cual puede ser accedido a través de protocolo SOAP.\\

Esta API posee las siguientes funcionalidades:

\begin{enumerate}
  \item Permite subir código fuente y compartirlo con otros usuarios.
  \item Permite ejecutar el código fuente con una data inicial en el lado servidor y en más de 60 lenguajes de programación diferentes.
  \item Permite descargar los resultados obtenidos en la ejecución del código fuente(salida, errores, información de compilación, tiempo de ejecución, uso de memoria, etc).
\end{enumerate}
\begin{figure}[h]
  \centering
  % Requires \usepackage{graphicx}
  \includegraphics[scale=0.5]{figuras/ideone.jpg}\\
  \caption{Ideone}\label{fig:ideone}
\end{figure}





\section{Algoritmos y programación}
\subsection{Estructura de datos}
\subsection{Búsqueda y Ordenamiento}
\subsection{Programación orientada a objetos}
