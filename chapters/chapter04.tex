\chapter{Aporte práctico}
En este capítulo se describe el sistema web que se va a implementar, detallando las funcionalidades que brindará el mismo, y también se describe la metodología de desarrollo con la cual será desarrollado el sistema.
\section{Las mejores prácticas de RUP}
EL Proceso Unificado Rational(RUP) es un proceso de ingeniería de software que provee un enfoque para la asignación de tareas y responsabilidades durante el desarrollo de un software. Tiene como objetivo asegurar la producción de un producto software de alta calidad que satisfaga los requerimientos de los usuarios finales dentro de un tiempo y presupuesto establecido \cite{rup_ibm_2014}.\\

Así mismo, RUP también es una guía para usar de manera efectiva el Lenguaje Unificado de Modelado (UML) que no es más que un lenguaje estándar que permite comunicar claramente los requerimientos, arquitecturas y diseños \cite{rup_ibm_2014}.\\

Es por ello que el desarrollo del sistema propuesto por esta tesis estará guiado por la buenas prácticas de RUP, estableciéndose iteraciones semanales en las cuales se irán desarrollando cada una de las fases en las que RUP divide el ciclo de desarrollo de software: Inicio, Elaboración, Construcción y Transición. También, se irá generando algunos de los artefactos según las disciplinas que RUP establece en su proceso de desarrollo.\\

A continuación se presenta un listado con los artefactos que serán entregados durante el proceso de desarrollo del sistema web de tiempo real para el aprendizaje colaborativo propuesto en esta tesis.

\begin{longtable}{|L{6cm}|L{9cm}|}
\caption{Artefactos del proceso de desarrollo}
\label{tab:artefactos_rup}\\
    \hline
    DISCIPLINA RUP & ARTEFACTO \\
    \hline
    Requisitos & $\bullet$ Modelo de caso de uso\\
    \hline
    \multirow{2}{*}{Análisis y diseño} & $\bullet$ Modelo de análisis\\
    \hhline{~~} & $\bullet$ Modelo de datos\\
    \hhline{~~} & $\bullet$ Prototipo de interfaz de usuario\\
    \hline

\end{longtable}






\section{Implementación de un sistema web de tiempo real para el aprendizaje colaborativo}

\section{}