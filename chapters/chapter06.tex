\chapter{Análisis de resultados}
En esta sección se van a aplicar las métricas de calidad definidas en la \autoref{sec:metricas_calidad} al Sistema Jigsaw Coding desarrollado en esta tesis.

\section{Métrica de calidad: Entendibilidad}
Aplicando la métrica definida en la \autoref{tab:c4_entendibilidad} se obtuvo los siguientes resultados:

\begin{longtable}{lc}
	\caption{Resultados obtenidos al aplicar la métrica de entendibilidad al Sistema Jigsaw Coding}
	\label{tab:resultados_metrica_entendibilidad}\\
	\toprule[0.7mm]
	\multicolumn{2}{c}{\emph{¿Qué tan fácil o difícil le resultó entender las funcionalidades del sistema?}}\\
	\midrule
	\textbf{Alternativa} & \textbf{Cantidad de votos} \\
	\midrule
	Muy fácil & 5 \\
	Fácil & 1 \\
	Ni fácil ni difícil & 0 \\
	Difícil & 0 \\
	Muy difícil & 0 \\
	\midrule
	Métrica & $x = \frac{A}{B} = \frac{5}{6} = 0.83$		\\
	\multicolumn{2}{l}{A : Cantidad de usuarios que selecionaron la alternativa Muy fácil.}\\
	\multicolumn{2}{l}{B : Cantidad de usuarios encuestados.}\\
	\bottomrule[0.7mm]
\end{longtable}

En la \autoref{tab:resultados_metrica_entendibilidad} se puede observar que el resultado de la métrica es 0.83, esto quiere decir que un 83\% de los usuarios del caso de estudio consideraron que las funcionalidades del Sistema Jigsaw Coding fueron muy fáciles de entender.

\section{Métrica de calidad: Portabilidad}
Aplicando la métrica definida en la \autoref{tab:c4_portabilidad} se obtuvo los siguientes resultados:

\begin{longtable}{lc}
	\caption{Resultados obtenidos al aplicar la métrica de portabilidad al Sistema Jigsaw Coding}
	\label{tab:resultados_metrica_portabilidad}\\
	\toprule[0.7mm]
	\multicolumn{2}{c}{\emph{¿En cuántos navegadores web puede usarse el sistema sin tener problemas?}}\\
	\midrule
	\textbf{Navegador} & \textbf{Hubo problemas} \\
	\midrule
	safari 5.1 &  NO \\
	opera 25 & NO \\
	firefox 33 & SÍ \\
	internet explorer 11 & NO \\
	chrome 39 & NO \\
	\midrule
	Métrica & $x = n = 4$		\\
	\multicolumn{2}{l}{n : Número de navegadores web en los que el sistema funciona correctamente.}\\
	\bottomrule[0.7mm]
\end{longtable}

En la \autoref{tab:resultados_metrica_portabilidad} se puede observar que el resultado de la métrica es 4, lo que indica que el Sistema Jigsaw Coding funciona correctamente en 4 navegadores web, los cuales funcionan en sistemas operativos Windows, Linux o IOS. Al aplicar la métrica se obtuvo que el Sistema Jigsaw Coding presenta incompatibilidad con el navegador Firefox en todas sus versiones, por lo que se recomienda usar cualquier otro navegador web.
\clearpage
\section{Métrica de calidad: Eficiencia}
Aplicando la métrica definida en la \autoref{tab:c4_eficiencia} se obtuvo los siguientes resultados:

\begin{longtable}{L{6cm}C{6cm}}
	\caption{Resultados obtenidos al aplicar la métrica de eficiencia al Sistema Jigsaw Coding}
	\label{tab:resultados_metrica_eficiencia}\\
	\toprule[0.7mm]
	\multicolumn{2}{c}{\emph{¿Cuál es el promedio de tiempo de respuesta del Sistema Jigsaw Coding?}}\\
	\midrule
	\textbf{Funcionalidad} & \textbf{Tiempo de respuesta(ms)} \\
	\midrule
	Generar grupos & 445\\
	Unirse a reunión de expertos & 1080\\
	Unirse a reunión jigsaw & 459\\
	Rendir examen & 594\\
	Resolver problema (Compilar) & 4016 \\
	\midrule
	\multirow{3}{*}{ Métrica } 
	& $x = \frac{\sum_{i}^{n}{t_{i}}}{n}, i=1,2,3,4,5 $	\\
	& $ x = \frac{445+1080+459+594+4016}{5}$ \\
	& $ x = 1318.8 ms$ \\
	\multicolumn{2}{l}{$t_{i}$ : Tiempo de respuesta en milisegundos de la funcionalidad $i$.}\\
	
	\bottomrule[0.7mm]
\end{longtable}

En la \autoref{tab:resultados_metrica_eficiencia} se puede observar que el resultado de la métrica es 1318.8 ms lo que es equivalente a decir que el tiempo de respuesta promedio del Sistema Jigsaw Coding es de 1.3 segundos. \\

En resumen, luego de aplicar las métricas definidas en la \autoref{sec:metricas_calidad} se obtuvieron los siguientes resultados:

\begin{itemize}
	\item El Sistema Jigsaw Coding es muy fácil de entender para el 83\% de los usuarios del caso de estudio.
	\item El Sistema Jigsaw Coding es compatible y funciona correctamente en 4 navegadores web: Chrome, Internet Explorer, Ópera y Safari.
	\item El Sistema Jigsaw Coding posee un tiempo promedio de respuesta de 1.3 segundos para las funcionalidades más importantes.
\end{itemize}

