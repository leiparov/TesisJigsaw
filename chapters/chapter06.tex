\chapter{Análisis de resultados}
\label{cap:analisis_de_resultados}
En esta sección se presentará las soluciones que se dieron a los problemas definidos en la \autoref{sec:problemas} y que fueron aplicados durante la sesión jigsaw. Además, se aplicarán las métricas de calidad (definidas en la \autoref{sec:metricas_calidad}) al Sistema Jigsaw Coding desarrollado en esta tesis y se mostrarán los resultados obtenidos.\\

\section{Soluciones a los problemas de la sesión jigsaw}
Durante la fase de expertos se plantearon los problemas \emph{Múltiplos de 7, Par o impar, Números descendentes} y a continuación se presentan las soluciones de los alumnos.

\begin{center}
	GE\_DÁT\_ECM $\longrightarrow$ Múltiplos de 7.
\end{center}

Elaborar un programa que imprima los primeros 10 números múltiplos de 7 usando la estructura repetitiva \texttt{for}.

\lstset{language=C, breaklines=true, basicstyle=\footnotesize}
\begin{lstlisting}
# include <iostream>
using namespace std;
int main (){
	int i;
	cout<<"Los 10 primeros mutiplos de 7 son:"<<endl;
	for(i=1;i<=10;i++){
		cout<<i*7<<endl;
	}
}
\end{lstlisting}
\clearpage
\begin{center}
	GE\_KMM\_DEG $\longrightarrow$ Par o impar.
\end{center}

Elaborar un programa que lea un entero y determine si es un número par o impar.

\lstset{language=Java, breaklines=true, basicstyle=\footnotesize}
\begin{lstlisting}
import java.util.Scanner;

class Ejercicio01 {
	public static void main (String args[] ){
		Scanner sc = new Scanner(System.in);

		System.out.print("Introduzca un numero entero: ");
		int n = sc.nextInt();

		if(n%2==0) System.out.print("El numero es par");
		else System.out.print("El numero es impar");
	}
}
\end{lstlisting}

\begin{center}
	GE\_SMM\_MRB $\longrightarrow$ Números descendentes.
\end{center}

Elaborar un programa que imprima de forma descendente los primeros 20 numeros pares usando una estructura repetitiva \texttt{while}.

\lstset{language=C, breaklines=true, basicstyle=\footnotesize}
\begin{lstlisting}
#include <iostream>
using namespace std;
int main (){
	int var = 40;
	while(var > 1){
		var = var - 2;
		cout<<var<<endl;
	}
	return 0;
}
\end{lstlisting}
\vspace{0.5cm}

Cuando la fase de expertos finalizó, cada alumno fue reagrupado para dar inicio a la reunión jigsaw y en ella cada grupo resolvió los 3 problemas: \emph{Múltiplos de 7, Par o impar, Números descendentes}. Estas soluciones se encuentran en la \autoref{tab:c6_soluciones_fase_jigsaw}
\clearpage
%\newgeometry{left=2cm,bottom=4cm, right=0.2cm}
\begin{landscape}

\begin{longtable}{cL{8cm}L{9cm}}
	\caption{Soluciones de fase jigsaw}
	\label{tab:c6_soluciones_fase_jigsaw}\\
	\toprule
	PROBLEMA & Solución del grupo GJ\_SMM\_KMM\_DÁT & Solución del grupo GJ\_MRB\_DEG\_ECM\\
	\midrule
	Múltiplos de 7 & 
	\begin{lstlisting}[language = C++, basicstyle=\footnotesize] 
	#include <iostream>
	using namespace std;
	int main (){
		int a = 7;
		for(int i=1; i<=10; i++){
			cout<<a<<endl;
			a = a + 7;
		}
		return 0;
	}		
	\end{lstlisting}&
	\begin{lstlisting}[language = C++, basicstyle=\footnotesize] 
	#include<iostream>
	using namespace std;
	int main()	{
		int i;	
		for(i=1;i<11;i++){
			cout<<i*7<<endl;
		}
	}		
	\end{lstlisting}\\
	%\midrule
	Par o impar &
	\begin{lstlisting}[language = C++, basicstyle=\footnotesize] 
	#include <iostream>
	using namespace std;
	int main (){
		int var; cin>>var;
		if(var%2==0) cout<<"Es par";
		else cout<<"es impar";
		return 0;
	}	
	\end{lstlisting}&
	\begin{lstlisting}[language = C++, basicstyle=\footnotesize] 
	#include <iostream>
	using namespace std;
	int main () {
		int a; cin>>a;
		if (a%2 == 0) cout<<"Es par"<<endl;
		else cout<<"Es impar"<<endl;
		return 0;
	}	
	\end{lstlisting}\\
	%\midrule
	Números descendentes & 
	\begin{lstlisting}[language = C++, basicstyle=\footnotesize] 
	#include <iostream>
	using namespace std;
	int main (){
		int var = 40;
		while(var>1){
			var = var - 2;
			cout<<var<<endl;
		}
		return 0;
	}
	\end{lstlisting}&
	\begin{lstlisting}[language = C++, basicstyle=\footnotesize] 
	#include<iostream>
	using namespace std;
	int main()	{
		int i=20;
		while(i>=1)	{
			cout<<2*i<<endl;
			i--;
		}
	}
	\end{lstlisting}\\
	%\bottomrule
\end{longtable}
\end{landscape}
%\restoregeometry
\section{Métrica de calidad: Entendibilidad}
Aplicando la métrica definida en la \autoref{tab:c4_entendibilidad} se obtuvo los siguientes resultados:

\begin{longtable}{lc}
	\caption{Resultados de la métrica de entendibilidad.}
	\label{tab:resultados_metrica_entendibilidad}\\
	\toprule[0.7mm]
	\multicolumn{2}{c}{\emph{¿Qué tan fácil o difícil le resultó entender las funcionalidades del sistema?}}\\
	\midrule
	\textbf{Alternativa} & \textbf{Cantidad de votos} \\
	\midrule
	Muy fácil & 1 \\
	Fácil & 3 \\
	Ni fácil ni difícil & 2 \\
	Difícil & 0 \\
	Muy difícil & 0 \\
	\midrule
	Métrica & $x = \frac{A}{B} = \frac{4}{6} = 0.66$		\\
	\multicolumn{2}{l}{A : Cantidad de usuarios que selecionaron la alternativa Fácil o Muy fácil.}\\
	\multicolumn{2}{l}{B : Cantidad de usuarios encuestados.}\\
	\bottomrule[0.7mm]
\end{longtable}

En la \autoref{tab:resultados_metrica_entendibilidad} se observa que el resultado de la métrica es 0.66, esto quiere decir que un 66\% de los usuarios del caso de estudio consideraron que las funcionalidades del Sistema Jigsaw Coding fueron muy fáciles de entender.

\section{Métrica de calidad: Portabilidad}
Aplicando la métrica definida en la \autoref{tab:c4_portabilidad} se obtuvo los siguientes resultados:

\begin{longtable}{lc}
	\caption{Resultados de métrica de portabilidad.}
	\label{tab:resultados_metrica_portabilidad}\\
	\toprule[0.7mm]
	\multicolumn{2}{c}{\emph{¿En cuántos navegadores web puede usarse el sistema sin tener problemas?}}\\
	\midrule
	\textbf{Navegador} & \textbf{Hubo problemas} \\
	\midrule
	safari 5.1 &  NO \\
	opera 25 & NO \\
	firefox 33 & SÍ \\
	internet explorer 11 & NO \\
	chrome 39 & NO \\
	\midrule
	Métrica & $x = n = 4$		\\
	\multicolumn{2}{l}{n : Número de navegadores web en los que el sistema funciona correctamente.}\\
	\bottomrule[0.7mm]
\end{longtable}

En la \autoref{tab:resultados_metrica_portabilidad} se observa que el resultado de la métrica es 4, lo que indica que el Sistema funciona correctamente en 4 navegadores web, los cuales pueden usarse en sistemas operativos Windows, Linux o Apple.
%\clearpage
\section{Métrica de calidad: Eficiencia}
Aplicando la métrica definida en la \autoref{tab:c4_eficiencia} se obtuvo los siguientes resultados:

\begin{longtable}{L{6cm}C{6cm}}
	\caption{Resultados de la métrica de eficiencia.}
	\label{tab:resultados_metrica_eficiencia}\\
	\toprule[0.7mm]
	\multicolumn{2}{c}{\emph{¿Cuál es el promedio de tiempo de respuesta del Sistema Jigsaw Coding?}}\\
	\midrule
	\textbf{Funcionalidad} & \textbf{Tiempo de respuesta(ms)} \\
	\midrule
	Generar grupos & 445\\
	Unirse a reunión de expertos & 1080\\
	Unirse a reunión jigsaw & 459\\
	Rendir examen & 594\\
	Resolver problema (Compilar) & 4016 \\
	\midrule
	\multirow{3}{*}{ Métrica } 
	& $x = \frac{\sum_{i}^{n}{t_{i}}}{n}, i=1,2,3,4,5 $	\\
	& $ x = \frac{445+1080+459+594+4016}{5}$ \\
	& $ x = 1318.8 ms$ \\
	\multicolumn{2}{l}{$t_{i}$ : Tiempo de respuesta en milisegundos de la funcionalidad $i$.}\\
	
	\bottomrule[0.7mm]
\end{longtable}

En la \autoref{tab:resultados_metrica_eficiencia} se puede observar que el resultado de la métrica es 1318.8 ms lo que es equivalente a decir que el tiempo de respuesta promedio del Sistema Jigsaw Coding es de 1.3 segundos. \\

En resumen, luego de aplicar las métricas definidas en la \autoref{sec:metricas_calidad} se obtuvieron los siguientes resultados:

\begin{itemize}
	\item El Sistema es muy fácil de entender para el 66\% de los usuarios del caso de estudio.
	\item El Sistema es compatible y funciona correctamente en 4 navegadores web: Chrome, Internet Explorer, Ópera y Safari.
	\item El Sistema posee un tiempo promedio de respuesta de 1.3 segundos para las funcionalidades más importantes.
\end{itemize}

