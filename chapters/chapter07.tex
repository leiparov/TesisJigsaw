\chapter{Conclusiones y Trabajos futuros}
\section{Conclusión general}
El objetivo general de esta tesis fue desarrollar un sistema web para promover el aprendizaje colaborativo de algoritmos y programación entre los estudiantes haciendo uso de la técnica de Jigsaw. \\

Este objetivo se ha cumplido satisfactoriamente al presentar el Sistema Jigsaw Coding, el mismo que ha sido expuesto en el \autoref{cap:aporte_practico} y que se encuentra alojado en la siguiente url: \url{www.jigsawcoding.com} 
\section{Conclusiones específicas}
\begin{itemize}
	\item En el \autoref{cap:introduccion} se desarrolló los antecedentes del problema y de la técnica que se usaría en esta tesis. Así mismo, se planteó la justificación de esta tesis y se definieron cuáles serían los objetivos y alcances de este trabajo de investigación.
	\item En el \autoref{cap:marco_teorico} se detalló algunos conceptos teóricos que permiten comprender mejor el contenido de esta tesis. Se definió qué es aprendizaje colaborativo y cuáles son sus beneficios al aplicarlo en los estudiantes.
	\item En el \autoref{cap:estado_del_arte} se plasmó el Estado del arte de las algunas herramientas y técnicas que permiten el aprendizaje colaborativo. 
	\item En el \autoref{cap:aporte_practico} se describió todo el desarrollo del Sistema Jigsaw Coding, para el cual se usó las mejores prácticas de RUP. Además, también se detalló el uso del sistema y se definieron 3 métricas de calidad que posteriormente serían aplicadas al sistema.
	\item En el \autoref{cap:caso_de_estudio} se presentó el caso de estudio sobre el cual se aplicaría el Sistema Jigsaw Coding. Se contó con las participación de 34 estudiantes universitarios de las dos escuelas (Sistemas y Software).
	\item Finalmente, en el \autoref{cap:analisis_de_resultados} se aplicaron al Sistema Jigsaw Coding las métricas de calidad definidas y se presentaron los resultados obtenidos. Así mismo, se presentó los resultados de las evaluaciones revelándose que aquellos alumnos que participaron de la Sesión Jigsaw obtuvieron un mejor rendimiento en su examen.
\end{itemize}

\section{Trabajos futuros}
Los trabajos futuros que se pueden considerar son los siguientes:

\begin{itemize}
	\item Agregar al Sistema Jigsaw Coding la funcionalidad de poder realizar conversaciones de audio con los demás miembros de grupos, pues con ello se tendrá una mejor comunicación entre ellos.
	\item Clasificar los problemas según el tipo de tema y permitir que el examen se genere aleatoriamente según el tópico que se esté desarrollando en la sesión jigsaw.
	\item Aplicar a los alumnos un test para determinar su estilo de aprendizaje y generar los grupos expertos y grupos jigsaw según dichos estilos.
\end{itemize}